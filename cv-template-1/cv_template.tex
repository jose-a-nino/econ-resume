%%%%%%%%%%%%%%%%%%%%%%%%%%%%%%%%%%%%%%%%%
% Important note:
% This template requires the res.cls file to be in the same directory as the
% .tex file. The res.cls file provides the resume style used for structuring the
% document.
%
%%%%%%%%%%%%%%%%%%%%%%%%%%%%%%%%%%%%%%%%%

%----------------------------------------------------------------------------------------
%	PACKAGES AND OTHER DOCUMENT CONFIGURATIONS (DO NOT CHANGE)
%----------------------------------------------------------------------------------------

\documentclass[margin, 11pt]{res} % Use the res.cls style, do not change font size at 11pt

\usepackage{helvet} % Default font is the helvetica postscript font
% \usepackage{newcent} % To change the default font to the new century schoolbook postscript font uncomment this line and comment the one above


\linespread{1.0}
\usepackage[T1]{fontenc}

\usepackage{fancyhdr}
\pagestyle{plain}
\cfoot{\thepage}


\usepackage{color}
\usepackage[colorlinks=true,
            linkcolor=redkey
            urlcolor=blue,
            citecolor=gray]{hyperref}

% Change margins and width (DO NOT CHANGE)

\setlength{\textwidth}{5.8in} % Text width of the document
\setlength{\textheight}{8 in} % increase text height to fit on 1-page 
\addtolength{\oddsidemargin}{-.3in}
\addtolength{\evensidemargin}{-.875in}
%	\addtolength{\textwidth}{1.75in}
\addtolength{\topmargin}{-.3in}
\addtolength{\textheight}{1.5in}

\hangindent=1em

\begin{document}

%----------------------------------------------------------------------------------------
%	NAME AND ADDRESS SECTION
%----------------------------------------------------------------------------------------


\name{\bf{\huge{José A. Niño}}\\[14pt]}  


\vspace{5cm}

% \address{Department of Economics\\ University of Houston \\
% 201 McElhinney Hall\\ Houston, TX 77204-5019 }

\address{ 
	E-mail:  \href{mailto:j.nino506@gmail.com}{\nolinkurl{j.nino506@gmail.com}} \\   
	Web: \url{[insert my website]} \\
	Mobile: 641-260-6902 \\
}   



%----------------------------------------------------------------------------------------

\begin{resume}



%----------------------------------------------------------------------------------------
%	EDUCATION
%----------------------------------------------------------------------------------------



\section{\sc{Education}}
% University of Houston,  \  Ph.D., Economics, 2015 (expected)\\
% \hspace*{5mm}Job Market Paper: \textit{This is the Title of Your Job Market}\\ 
% \hspace*{5mm}Dissertation Committee:  First Advisor, Second Advisor, Third Advisor
% or use the following line
%\hspace*{5mm}Main Advisor:  First Advisor

% University of Houston,  \  M.A., Economics, 2012\\
Kenyon College,  \  B.A., Economics \& Mathematics, May 2023 (\emph{Magna Cum Laude}) 


%----------------------------------------------------------------------------------------
%	RESEARCH
%----------------------------------------------------------------------------------------


\section{\sc{Fields of\\ Interest}}
Macroeconomics, Econometrics, Labor Economics \\


% \section{\sc{Publications}}
% Smith, Madeline T. 2014. ``This is the Title of My Paper.'' \emph{Journal of Human Resources}, 50(3): 124-150.\\
 

% \section{\sc{Working\\ Papers} }
% % Means ready to share draft

% ``This is the Title of Your Job Market Paper.'' \textbf{[Job Market Paper]} 

% \emph{Abstract:} This is the abstract.\\

% ``This is an Another Title of Your Working Paper,"  November 2014. 

% ``This is an Another Title of Your Working Paper" (with Co-Author Name), November 2014.\\


% \section{\sc{Works in\\ Progress} }
% % Do not need a draft but can describe basic objective; no date needed

% "This is the Title of Your Work in Progress"

% "This is the Title of Your Work in Progress" (with Co-Author Name)

% "This is the Title of Your Work in Progress"\\



%----------------------------------------------------------------------------------------
%	EXPERIENCE
%----------------------------------------------------------------------------------------

% \newpage % Delete depending on your length of CV

\section{\sc{Relevant Experience}}


\textbf{Research Experience}

% \begin{itemize}
% 	\item Research Assistant, Federal Reserve Board of Governors, \emph{Monetary Studies}, July 2022 - Present 
% 	\item Full-Year Intern, Federal Reserve Board of Governors, \emph{Policy Analytics}, October 2022 - July 2022 
% 	\item Summer Intern, Federal Reserve Board of Governors, \emph{Capital Markets}, June - August 2022
% \end{itemize}	

Research Assistant, Federal Reserve Board of Governors, Monetary Studies \\ 
July 2022 - Present
	\begin{itemize}
		\item  
		\item  
		\item 
	\end{itemize}

Full-Year Intern, Federal Reserve Board of Governors, \emph{Policy Analytics}
\\ 
October 2022 - July 2022 
	\begin{itemize}
		\item Created templates in Microsoft Excel to test banks on their compliance with the Volcker Rule and Basel III banking standards
		\item Wrote a Python script to clean city emissions data between 2012 through 2022 for Li Gu
		\item Created a table using Shiny and R to an existing dashboard to detail different bank's capital gains/losses and compliance with Basel III banking standards and the Volcker rule
		\item Curated documentation on future features to add to the dashboard for future use by the Policy Analytics section using RMarkdown and Git
	\end{itemize}
Summer Intern, Federal Reserve Board of Governors, Capital Markets \\ June - August 2022 
	\begin{itemize}
		\item Translated SAS code into R code to replicate debt ratio charts on Capital Markets (CM) internal website 
		\item Developed four new market leverage charts for the CM section using R and Compustat data
		\item Created and updated datasets of leveraged buyouts and special purpose acquisition companies for future research by merging data in from Compustat and CRSP 
		\item Wrote documentation on code for future use by the CM section using RMarkdown and Git
		\item Participated in PSS R summer school learning advanced skills in R and completing assignments in functions, data visualization, and data cleaning 
	\end{itemize}

Scholar, Howard University, American Economic Association Summer Program \\
June - July 2021
	\begin{itemize}
		\item Selected to participate in summer intensive program as one of 28 scholars
		\item Excelled in rigorous coursework in math, microeconomics, and econometrics with 4.0 GPA
		\item Attended seminars and panels on career development, graduate school, and research opportunities in the economics field
		\item Collaborated with other scholars on research projects, presentations, and problem sets
		\item Wrote econometrics project on determinants of individual public health insurance in 2019 leveraging a linear probability model and IPUMS data
		\item Utilized R to complete econometrics problem sets and \LaTeX to write assignments up in a professional format
	\end{itemize}

Research Intern, Bureau of Labor Statistics \\
June - July 2021
	\begin{itemize}
		\item Collaborated weekly with economists at the Bureau of Labor Statistics
		\item Created multiple visualizations of Employment Cost Index (ECI) data using ggplot2 from R
		\item Assisted in writing a report detailing the findings of relative importance measure in relation to ECI
	\end{itemize}
\textbf{Teaching Experience}

	\emph{Lead Tutor},  Calculus III, Kenyon College, Fall 2022 \& Spring 2023
		\begin{itemize}
			\item 
		\end{itemize}

	\emph{Project Buddy}, ECON 181, Howard University, Fall 2023

	\emph{Lead TA}, Advanced Research Methods (AEASP), Howard University, Summer 2023

%----------------------------------------------------------------------------------------
%	OTHER
%----------------------------------------------------------------------------------------


\section{\sc{Fellowships \\ and Awards}}
Risk-Taker Award, Kenyon College, April 2022 \\
Merit List, Kenyon College, Fall 2019 - Spring 2023 \\
Music Scholarship Recipient, Kenyon College, Spring 2019 (\$60,000) 


\section{\sc{Presentations}}
American Economic Association Summer Program, July 2021 \\
American Economic Association Summer Pipeline Conference, June 2021 \\


% \section{\sc{Refereeing Experience}}
% \emph{Journal of Labor Economics}, \emph{Journal of Human Resources}\\

\section{\sc{Computer\\ Skills}}
R, Python, Matlab, Linux, C\texttt{++}, \LaTeX \\
 
\section{\sc{Affiliations}}
American Economic Association, American Society of Hispanic Economics

\section{\sc{Languages}}
English (Native), Spanish (Advanced)

\section{\sc{Citizenship /\\ Visa}}
U.S.A.\\



%----------------------------------------------------------------------------------------
%	References
%----------------------------------------------------------------------------------------


\section{\sc{References}}

% \vspace{.05in}
% \begin{tabular}{@{}p{3in}p{4in}}
% \textbf{Professor Chinhui Juhn}   						& \textbf{Professor Aimee Chin} \\            
% Department of Economics   							& Department of Economics \\         
% University of Houston 								& University of Houston\\       
% Phone: (617) 353-4824 								& Phone: (617) 353-5694\\    
% E-mail:  \href{mailto:cjuhn@uh.edu}{\nolinkurl{cjuhn@uh.edu}}& E-mail:  \href{mailto:achin@uh.edu}{\nolinkurl{achin@uh.edu}}
% \end{tabular}

% \vspace{.05in}
% \begin{tabular}{@{}p{3in}p{4in}}
% \textbf{Professor Chinhui Juhn}   						& \textbf{Professor Aimee Chin (Teaching)} \\            
% Department of Economics   							& Department of Economics \\         
% University of Houston 								& University of Houston\\       
% Phone: (617) 353-4824 								& Phone: (617) 353-5694\\    
% E-mail:  \href{mailto:cjuhn@uh.edu}{\nolinkurl{cjuhn@uh.edu}}& E-mail:  \href{mailto:achin@uh.edu}{\nolinkurl{achin@uh.edu}}
% \end{tabular}


%----------------------------------------------------------------------------------------

\end{resume}
\end{document}